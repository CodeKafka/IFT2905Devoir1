\documentclass[9pt]{report}
%\usepackage[utopia]{mathdesign} 

\usepackage{amsmath,amsfonts,amsthm,amssymb,mathtools}
%\usepackage{amsmath,amsthm,mathtools}
%\usepackage{libertine}
\input{preamble.tex}
\usepackage[scr]{rsfso}


\lstset{basicstyle=\ttfamily}
%\usepackage[euler-digits]{eulervm}
\usepackage{mathpazo}
%usepackage{palatino}
%usepackage{crimson}


\title{\Huge{Interface PM }\\{IFT2015}\\{\textbf{Devoir 1}}}
\author{\huge{Franz Girardin}}
\date{\today}
\lstset{inputencoding=utf8/latin1}






            %%%%%%%%%%%%%%%%%  Sect.                          %%%%%%%%%%%%%%%%%%%%%%%%%%%%%%%%%%%%%%%%%%%%%%%%%%%%%%%%%
\usepackage{helvet}
\titleformat{\chapter}
  {\fontfamily{}\bfseries\huge} % format
  {}                % label
  {0pt}             % sep
  {\color{myb}\huge}           % before-code



\titleformat{\section}
  {\normalfont\scshape}{\thesection}{1em}{}


% Customizing the spacing for the chapter titles
\titlespacing*{\chapter}{0pt}{0pt}{20pt}

% Allow hfill in math environment
\newcommand{\specialcell}[1]{\ifmeasuring@#1\else\omit$\displaystyle#1$\ignorespaces\fi}

% Allow you to do the non implication (implication barred)
\newcommand{\notimplies}{%
  \mathrel{{\ooalign{\hidewidth$\not\phantom{=}$\hidewidth\cr$\implies$}}}}



\DeclareRobustCommand{\looongrightarrow}{%
  \DOTSB\relbar\joinrel\relbar\joinrel\relbar\joinrel\rightarrow
}


\DeclareMathOperator{\di}{d\!}
\newcommand*\Eval[3]{\left.#1\right\rvert_{#2}^{#3}}

\begin{document}
\maketitle

\pagebreak
\tableofcontents
\pagebreak


\chapter{Scénario }
  \vspace{-2em}
\begin{multicols*}{2}
  \small
  \begin{Exercice}{(5 pts)}{}
    Écrivez un \textbf{scénario}, c’est-à-dire une séquence d’actions sur 
    la façon dont vous utiliseriez cette
    interface
  \end{Exercice}


  \section{Persona}
  \textbf{Zephyra} est une jeune ingénieure talentueuse de 23 ans, 
  évoluant dans le monde dynamique du développement de jeux vidéo. 
  Lors d'une conversation détendue avant une réunion d'équipe, 
une de ses collègues lui parle \textit{d'un site Web} récemment découvert.

  \textbf{Améthiste}, une femme accomplie aux portes de la retraite, 
  occupe un rôle clé en tant que gestionnaire des ressources humaines. 
  Zephyra, bien consciente de l'intérêt d'Améthiste pour la spiritualité, 
  les pratiques tantriques, la découverte des chakras et l'astrologie, 
  est elle-même attirée par ces sujets. Cependant, elle est initialement 
  sceptique quant à l'utilité d'un site Web dédié, préférant les sources 
  variées et interactives disponibles sur des plateformes telles que 
  \texttt{TikTok}.

  Soucieuse de tisser des liens amicaux avec sa supérieure et désireuse 
  de combler le fossé générationnel entre elles, Zephyra décide de donner 
  une chance à cette recommandation. À la fin de leur échange, elle exprime 
  à Améthiste sa curiosité grandissante et s'engage à explorer le site 
  pour partager ses impressions.


  \section{Scénarios et cas d'utilisation} 

  \paragraph{Visite initiale}
  Lors de sa première visite, Zephyra est attirée par le 
  \textit{design unique} du 
  site et commence par explorer les différentes \textbf{sections}
  présentées sur la page d'accueil. Elle s'intéresse particulièrement 
  aux \textit{lectures gratuites} marquées par le texte 
  \textcolor{red}{\texttt{FREE!}} et aux informations sur l'ascension, 
  parcourant les titres et les images pour se faire une idée 
  du contenu offert Elle constate rapidement la quantité impressionnante 
  de sections et de liens cliquable et tente de comprendre 
  la \textit{signification} de chacun d'eux.


  \begin{figure}[H]
    \begin{center}
      \includegraphics[width=0.45\textwidth]{NavigationInitialeNavbar.png}
    \end{center}
  \end{figure}


  \paragraph{Navigation}
  Navigation : Zephyra utilise la barre de navigation pour explorer 
  les différentes catégories du site. Elle clique sur des onglets et 
  sous-menus, découvrant divers services et contenus. 
  Elle passe du temps à examiner chaque section, essayant de 
  comprendre la structure du site et la relation entre les différentes 
  parties.

  \paragraph{Recherche d'information}
  Dans cette étape, 
  Zephyra se concentre sur la collecte d'informations spécifiques. 
  L'auteure du site, \textbf{Jami Lin}, semble avoir un système  
  compréhensif pour l'épanouissement personnel et l'éveil spirituel. 
  Le site comprend une grande quantité d'information liées à 
  sa \textit{philosophie}. 
  Zephyra clique donc sur des liens pour approfondir sa compréhension des 
  services et  informations, 
  lit des descriptions détaillées et regarde 
  éventuellement des vidéos ou des images associées pour enrichir 
  sa compréhension.

  \columnbreak


\begin{figure}[H]
 \dirtree{%
.1 Barre de Navigation.
.2 E-Learn.
.3 \entouree{\{ sous-options \}}.
.4 Law of attaction.
.5 \entoure{\{ liens \}}.
.6 \textit{Why doesn't 'Law of Attraction work' }. 
.6 \textit{Click Here ... Manifestation Mastery E-learn}.
.4 What is Galactic for.
.5 \entoure{\{ liens \}}.
.6 \textit{Why doesn't 'Law of Attraction work' }.
.4 $\dots$. 
.4 FREE! Treasure Chest.  
.5 \entoure{\{ liens \}}.
.6 \textit{Why doesn't 'Law of Attraction work' }.
.2 SIGN-UP.
.3 SIGN-IN. 
.4 Username.
.4 Password. 
.3 \entoure{\{ liens \}}.
.4 Forgot Password.
.4 \textit{Join Jami Live On E-Lean}.   
.2 TODAY's. 
.3 $\dots$.
.2 ACTIVATE.
.3 $\dots$.
.2 EVOLVE. 
.3 $\dots$.
.2 MASTER. 
.3 $\dots$.
.2 5D. 
.3 $\dots$.
.2 GUIDES. 
.3 $\dots$.
.2 STORE. 
.3 $\dots$.
} 
\caption{Aperçu de la barre de navigation}
\end{figure}


\begin{note}{}{}
    Chaque \textbf{onglet} de la barre de navigation comprends 
    au moins 4 \texttt{sous-options} organisées dans une menu défilant.   
\end{note}

  \paragraph{Inscription}
  Zephyra décide de s'inscrire pour accéder à plus de contenus. Elle 
  remplit le formulaire d'inscription, fournissant ses informations
  personnelles et créant un compte utilisateur. Elle prend le temps de 
  lire les termes et conditions et complète le processus d'inscription.



  \paragraph{Exploration du contenu personalité}
  Une fois inscrite, Zephyra explore les fonctionnalités personnalisées. 
  Elle navigue à travers les options disponibles pour 
  les utilisateurs enregistrés, découvrant des lectures personnalisées, 
  des activations et d'autres contenus exclusifs. Elle teste différentes 
  fonctionnalités pour voir comment elles répondent à ses intérêts.


  \paragraph{Recherche de support et contact}
  Zephyra cherche des informations supplémentaires ou de l'aide sur 
  l'utilisation du site. Elle navigue vers la section d'aide ou de 
  contact, où elle recherche des FAQ, des guides d'utilisation, ou 
  un moyen de contacter directement le support client ou l'auteur Jami Lin, 
  puisqu'il est mentionné plusieurs fois qu'il est possible de la 
  contacter directement par courriel. 



  




  \vspace{-1em}









\end{multicols*}
\end{document}
